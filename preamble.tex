% ----------------------------------------------------------------
% Preamble DE
% ---------------------------------------------------------------- 
\chapter*{Bachelorarbeit}
\textbf{Thema:}

Authentifizierungsmethoden für digitale Fernsignaturen auf Grundlage der Handy-Signatur
\\[4ex]
\textbf{Gutachter:}
\begin{enumerate}
    \item Prof. Dr.-Ing. Luigi Lo Iacono (Technische Hochschule Köln)
    \item Dipl.-Ing. Tobias Wagner (n-design GmbH)
\end{enumerate}
\textbf{Zusammenfassung:}

Was ein Abstract ist wird in der DIN Norm 1426 festgelegt: es ist ein Kurzreferat zur Inhaltsangabe. Die Definition des American National Standards Institute (ANSI) lautet: „An abstract is defined as an abbreviated accurate representation of the con-tents of a document". Es sollten 8-10 Zeilen Text folgen.
\\[4ex]
\textbf{Stichwörter:}

Stichwort, Stichwort (hier sollten maximal 5 Stichwörter folgen)
\\[4ex]
\textbf{Datum:}

\today
\clearpage

% ----------------------------------------------------------------
% Preamble EN
% ---------------------------------------------------------------- 
\chapter*{Bachelors Thesis}
\textbf{Title:}

Authentication methods for digital remote signatures based on mobile phone signatures
\\[4ex]
\textbf{Reviewers:}
\begin{enumerate}
    \item Prof. Dr.-Ing. Luigi Lo Iacono (Technische Hochschule Köln)
    \item Dipl.-Ing. Tobias Wagner (n-design GmbH)
\end{enumerate}
\textbf{Abstract:}

Hier sollte eine Übersetzung des obigen Abstracts auf Englisch erfolgen (und kein kom-plett neuer Text). Auch hier bitte die Begrenzung auf maximal 10 Zeilen Text einhalten.
\\[4ex]
\textbf{Keywords:}

(hier die Übersetzung der obigen Stichworte)
\\[4ex]
\textbf{Date:}

\today
\clearpage