\author{Philipp Dominik Bzdok}

\title{Bachelorarbeit Medientechnologie \linebreak \linebreak Authentifizierungsmethoden für elektronische Signaturen auf Grundlage der Handy-Signatur}

\documentclass[11pt,a4paper,ngerman]{report}
\usepackage[utf8]{inputenc}
\usepackage[T1]{fontenc}
\usepackage{fontspec}
\usepackage[ngerman]{babel}

\usepackage{color}
\usepackage{tabularx}
\usepackage{todonotes}
\usepackage{caption}

\usepackage[titles]{tocloft}
\newlistof{listing}{lol}{Quellcodeverzeichnis}

\usepackage[newfloat]{minted}
\usemintedstyle{xcode}
\setminted[java]{linenos,tabsize=2,breaklines,fontsize=\footnotesize,frame=lines,framesep=2ex,autogobble}
\setminted[ruby]{linenos,tabsize=1,breaklines,fontsize=\footnotesize,frame=lines,framesep=2ex,autogobble}

\usepackage{hyperref}
\hypersetup{pdfborder={0 0 0}, breaklinks=true}

\setlength{\parskip}{0.5em}
\setlength{\parindent}{0em}

\newcommand{\ruby}[1]{\mintinline{ruby}{#1}}

% !TEX root = Bachelorarbeit_Philipp_Bzdok

% ----------------------------------------------------------------
% Settings of Quellcodeverzeichnis
% ----------------------------------------------------------------
\setlength{\cftlistingnumwidth}{3em}
\begin{document}
\maketitle

% ----------------------------------------------------------------
% Preamble DE
% ---------------------------------------------------------------- 
\chapter*{Bachelorarbeit}
\textbf{Thema:}

Authentifizierungsmethoden für elektronische Signaturen auf Grundlage der Handy-Signatur
\\[4ex]
\textbf{Gutachter:}
\begin{enumerate}
    \item Prof. Dr.-Ing. Luigi Lo Iacono (Technische Hochschule Köln)
    \item Dipl.-Ing. Tobias Wagner (n-design GmbH)
\end{enumerate}
\textbf{Zusammenfassung:}

Was ein Abstract ist wird in der DIN Norm 1426 festgelegt: es ist ein Kurzreferat zur Inhaltsangabe. Die Definition des American National Standards Institute (ANSI) lautet: „An abstract is defined as an abbreviated accurate representation of the con-tents of a document". Es sollten 8-10 Zeilen Text folgen.
\\[4ex]
\textbf{Stichwörter:}

Stichwort, Stichwort (hier sollten maximal 5 Stichwörter folgen)
\\[4ex]
\textbf{Datum:}

\today
\clearpage

% ----------------------------------------------------------------
% Preamble EN
% ---------------------------------------------------------------- 
\chapter*{Bachelors Thesis}
\textbf{Title:}

Authentication methods for electronic signatures based on mobile phone signature
\\[4ex]
\textbf{Reviewers:}
\begin{enumerate}
    \item Prof. Dr.-Ing. Luigi Lo Iacono (Technische Hochschule Köln)
    \item Dipl.-Ing. Tobias Wagner (n-design GmbH)
\end{enumerate}
\textbf{Abstract:}

Hier sollte eine Übersetzung des obigen Abstracts auf Englisch erfolgen (und kein kom-plett neuer Text). Auch hier bitte die Begrenzung auf maximal 10 Zeilen Text einhalten.
\\[4ex]
\textbf{Keywords:}

(hier die Übersetzung der obigen Stichworte)
\\[4ex]
\textbf{Date:}

\today
\clearpage

% ----------------------------------------------------------------
% Table of contents
% ---------------------------------------------------------------- 
\tableofcontents
\clearpage

% ----------------------------------------------------------------
% Quellcodeverzeichnis
% ----------------------------------------------------------------
\phantomsection
\addcontentsline{toc}{chapter}{Quellcodeverzeichnis}
\listoflistings
\clearpage

% ----------------------------------------------------------------
% Content
% ----------------------------------------------------------------

\chapter{Einleitung}
Im sechsten Fachsemester des Studiengangs Medientechnologie an der Technischen Hochschule Köln ist eine Praxis- \& Mobilitätsphase für jeden Studierenden vorgesehen.
Dieses Semester kann relativ frei gestaltet werden, muss jedoch in einem fachverwandten Bereich absolviert werden und einen Umfang von 450 Stunden haben.
Jeder Studierende hat die Möglichkeit ein Praktikum an der Technischen Hochschule oder in einer externen Firma zu machen. Bei Zweiterem muss der Student sich jedoch selbst um einen Platz bemühen.

In meinem Fall ist die Entscheidung für ein externes Praktikum relativ früh gefallen, da ich seit dem 01. Februar 2018 bereits in der Kölner Firma \href{https://n-design.de/}{n-design GmbH} als Werkstudent im Bereich Software-Entwicklung angestellt bin. Die Firma n-design spezialisiert sich im Bereich ``e-Health''. Das aktuell wichtigste Projekt ist der \href{https://www.kococonnector.com/produkte/konnektor.html}{KocoConnector}, für den die Anwendungskomponente von n-design entwickelt wird:
\begin{quote}
    Die wichtigste Komponente in der Telematikinfrastruktur, der Konnektor als sicherer Zugangspunkt zu den Fachdiensten der Kostenträger, wurde durch die n-design mitentwickelt. Der Anwendungskonnektor, eine zentrale Teilkomponente des Konnektors, wurde vollständig durch n-design realisiert. Durch den Anwendungskonnektor wird die Kommunikation zu Clientsystemen und Kartenterminals, sowie auch die sichere Kommunikation in die Telematikinfrastruktur mittels Fachmodulen ermöglicht.
\end{quote}
Somit deckt der Konnektor viele Bereiche der Softwareentwicklung ab. Da das Projekt so vielseitig ist, gab es mir die Möglichkeit in den verschiedenen Bereichen wie \textit{Java}, \textit{embedded Systems}, \textit{Webschnittstellen}, \textit{Kryptographie} und \textit{GIT} zu arbeiten und mich dort weiterzubilden.

Im folgenden werde ich meine Erfahrungen und Erkenntnisse, welche ich während des Praktikums bei n-design als aktives Mitglied der Softwareentwicklung sammeln durfte, darstellen.

Test Ruby Code:
\begin{listing}
    \inputminted{ruby}{/Users/pbz/Documents/Projects/hasher_api/app/models/calculation.rb}
    \caption{Calculation.rb}
    \label{lst:Calculation}
\end{listing}

\end{document}