\author{Philipp Dominik Bzdok}

\title{Bachelorarbeit Medientechnologie \linebreak \linebreak
Authentifizierungsmethoden für digitale Fernsignaturen auf Grundlage der Handy-Signatur}

\documentclass[11pt,a4paper,ngerman]{report}
\usepackage[T1]{fontenc}
\usepackage{fontspec}
\usepackage[ngerman]{babel}

\usepackage{fancyhdr}
\pagestyle{fancy}
\fancyhf{}
\setlength{\headheight}{14pt}
\fancyhead[R]{\textsl{\rightmark}}
\fancyfoot[R]{\thepage}

\fancypagestyle{plain}{%
    \renewcommand{\headrulewidth}{0pt}%
    \fancyhf{}%
    \fancyfoot[R]{\thepage}%
}

\usepackage{color}
\usepackage{tabularx}
\usepackage{todonotes}
\usepackage[skip=0pt]{caption}

\usepackage[titles]{tocloft}
\newlistof{listing}{lol}{Quellcodeverzeichnis}

\usepackage[newfloat]{minted}
\usemintedstyle{xcode}
\setminted[java]{linenos,tabsize=2,breaklines,fontsize=\footnotesize,frame=lines,framesep=2ex,autogobble}
\setminted[ruby]{linenos,tabsize=1,breaklines,fontsize=\footnotesize,frame=lines,framesep=2ex,autogobble}

\usepackage{hyperref}
\hypersetup{pdfborder={0 0 0}, breaklinks=true}

\setlength{\parskip}{0.5em}
\setlength{\parindent}{0em}

\setlength{\cftlistingnumwidth}{3em}

\newcommand{\ruby}[1]{\mintinline{ruby}{#1}}

% !TEX root = Bachelorarbeit_Philipp_Bzdok

\begin{document}
\maketitle

% ----------------------------------------------------------------
% Preamble DE
% ---------------------------------------------------------------- 
\chapter*{Bachelorarbeit}
\textbf{Thema:}

Authentifizierungsverfahren für digitale Fernsignaturen auf Grundlage der Handy-Signatur
\\[4ex]
\textbf{Gutachter:}
\begin{enumerate}
    \item Prof. Dr.-Ing. Luigi Lo Iacono (Technische Hochschule Köln)
    \item Dipl.-Ing. Tobias Wagner (n-design GmbH)
\end{enumerate}
\textbf{Zusammenfassung:}

Die digitale Signatur stellt ein wichtiges Werkzeug der modernen Informationstechnik dar und durch Fernsignaturen ist es heutzutage möglich, digital, rechtsgültige Unterschriften zu leisten. Der Prozess der Signatur setzt dementsprechend eine starke Authentifizierung des Benutzers voraus. In diesem Kontext wird die SMS-TAN, das erste Authentifizierungsverfahren der österreichischen Handy-Signatur und das am weitesten verbreitete Zwei-Faktor Authentifizierungsverfahren, analysiert. Darauf aufbauend werden alternative Authentifizierungsverfahren für den Anwendungsbereich Fernsignatur evaluiert. Durch die prototypische Implementierung der TOTP und U2F Authentifizierungsverfahren wird gezeigt, wie mit überschaubarem Aufwand die unsichere SMS-TAN substituiert werden kann um somit ein höheres Sicherheitsniveau zu erreichen.
\\[4ex]
\textbf{Stichwörter:}

Authentifizierung, digitale Signatur, Kryptographieverfahren, SMS-TAN, TOTP, U2F
\\[4ex]
\textbf{Datum:}

\today
\clearpage

% ----------------------------------------------------------------
% Preamble EN
% ---------------------------------------------------------------- 
\chapter*{Bachelors Thesis}
\textbf{Title:}

Authentication procedures for digital remote signatures based on mobile phone signatures
\\[4ex]
\textbf{Reviewers:}
\begin{enumerate}
    \item Prof. Dr.-Ing. Luigi Lo Iacono (Technische Hochschule Köln)
    \item Dipl.-Ing. Tobias Wagner (n-design GmbH)
\end{enumerate}
\textbf{Abstract:}

The digital signature is an important tool of modern information technology and with remote signatures it is now possible to provide digital, legally valid signatures. Accordingly, the signature process requires a strong authentication of the user. In this context, the SMS-TAN, the first authentication procedure of the Austrian mobile phone signature and the most widespread two-factor authentication procedure, is analyzed. Based on this, alternative authentication methods for the field of remote signatures will be evaluated. Through the prototypical implementation of the TOTP and U2F authentication methods, it will be shown how the unsecure SMS-TAN can be substituted with a manageable effort in order to achieve a higher level of security.
\\[4ex]
\textbf{Keywords:}

Authentication, digital signature, cryptographic methods, SMS-TAN, TOTP, U2F
\\[4ex]
\textbf{Date:}

\today
\clearpage

% ----------------------------------------------------------------
% Eidesstattliche Erklärung
% ---------------------------------------------------------------- 
\chapter*{Eidesstattliche Erklärung}
Ich erkläre an Eides statt, dass ich die vorgelegte Abschlussarbeit selbständig und ohne fremde Hilfe verfasst, andere als die angegebenen Quellen und Hilfsmittel nicht benutzt und die den benutzten Quellen wörtlich oder inhaltlich entnommenen Stellen als solche kenntlich gemacht habe.

\vspace{1.5cm}

\_\_\_\_\_\_\_\_\_\_\_\_\_\_\_\_\_\_\_\_\_\_ \\
(Ort, Datum)

\vspace{0.5cm}

\_\_\_\_\_\_\_\_\_\_\_\_\_\_\_\_\_\_\_\_\_\_ \\
(Unterschrift)

\clearpage

% ----------------------------------------------------------------
% Vorwort
% ---------------------------------------------------------------- 
\chapter*{Vorwort}
Hier möchte ich auf die Umstände der Abschlussarbeit eingehen und denen danken, die mir dabei geholfen haben.
\clearpage

% ----------------------------------------------------------------
% Inhaltsverzeichnis
% ---------------------------------------------------------------- 
\tableofcontents
\clearpage

% ----------------------------------------------------------------
% Inhalt
% ----------------------------------------------------------------

\chapter{Einleitung}
Digitale Signaturen nehmen in der heutigen Zeit eine wichtige Rolle in der Informationstechnik und somit im Alltag ein. Mit Hilfe von digitalen Signaturen ist es möglich beliebigen Daten zweifelsfrei Urheberschaft und Integrität nachzuweisen. Die Eigenschaften der Urheberschaft und Integrität sind besonders wichtig im juristischen Kontext der elektronischen Signatur, in welchem sie das Gegenstück zur handschriftlichen Unterschrift bildet. Der Begriff elektronische Signatur wird oft synonym zur digitalen Signatur verwendet, jedoch ist die elektronische Signatur ein juristischer Begriff, welcher erstmals in einem überarbeiteten Entwurf der EU-Richtlinie 1999/93/EG von der EU-Kommission verwendet wurde \cite{eSigEU99}. Der Begriff digitale Signatur hingegen stammt aus der Mathematik beziehungsweise aus der Kryptographie und bezeichnet einerseits ein asymmetrisches Kryptographieverfahren, aber anderseits den mathematischen Wert welcher durch das Kryptographieverfahren erzeugt wird.

Digitale Signaturen werden bereits in einem breiten Spektrum von Anwendungen eingesetzt, wie zum Beispiel in PGP-Systemen\footnote{Pretty good Privacy - 1986 bis 1991 von Phil Zimmermann entwickelt.} oder Zertifikatsbasierten Systemen. Des weiteren werden digitale Signaturen immer häufiger in neuen Anwendungsbereichen eingesetzt. Eine besondere Anwendung auf die ich im Laufe dieser Ausarbeitung explizit eingehen möchte, ist die österreichische \textit{Handy-Signatur}. Die \textit{Handy-Signatur} ermöglicht es den Bürgern Österreichs sich ohne zusätzliche Hardware, wie Kartenlesegeräte, online per Handy auszuweisen oder Formulare und Dokumente elektronisch zu unterschreiben \cite{handySigOnline}. Die Besonderheit dieser digitalen Signatur ist, dass die \textit{Handy-Signatur} eine qualifizierte elektronische Signatur auslöst und somit juristisch einer handgeschriebenen Unterschrift gleichgesetzt werden kann. Die für die Signatur benutzte Authentifizierungsmethode ist die mobile \textit{SMS-TAN}. Diese Authentifizierungsmethode wurde jedoch bereits von \textit{NIST}\footnote{National Institute of Standards and Technology} in ihrer Tauglichkeit eingeschränkt, da SMS nie dafür gemacht worden ist Geheimnisse zu übertragen \cite{mobileSec,NIST800-63B}. Trotz der mangelhaften Eignung von SMS als Kanal für OOB-Authentifizierung\footnote{Out-Of-Band Authentifizierung - Authentifizierung über separaten Kanal}, ist das SMS-OTP\footnote{One time password} die am meisten verbreitete Variante der multifaktor Authentifizierung \cite{fido17}.

Aus diesem Grund möchte ich in dieser Arbeit die Eigenschaften der Authentifizierungsmethode der \textit{Handy-Signatur} beleuchten und darauf basierend Alternativen evaluieren und prototypisch implementieren. Mein Ziel ist es, die modernen Möglichkeiten von Multi-Faktor Authentifizierung im Kontext der digitalen Signatur zu Vergleichen und zu bewerten, um abschließend eine Aussage über die Tauglichkeit für den Anwendungsbereich \textit{Fernsignatur} machen zu können.
\clearpage


\chapter{Grundlagen}
Um die Eignung von Authentifizierungsmethoden bestimmen zu können, müssen erst einmal die Rahmenbedingung festgelegt werden. Die digitale Signatur und die \textit{Handy-Signatur} werden im folgenden beleuchtet, da dies der Anwendungsbereich ist, welchen die Authentifizierungsmethoden schützen sollen. Dazu stelle ich verschiedenste Signaturvarianten und ihre schützenswerten Merkmale dar. Nach der Bestimmung dieser Eigenschaften wird die Authentifizierung im Kontext der Signatur erläutert.

\section{Digitale Signaturverfahren}
Als \textit{digitales Signaturverfahren} bezeichnet man im allgemeinen ein asymmetrisches Kryptosystem\footnote{Oberbegriff für Public-Key-Verschlüsselungsverfahren}

% ----------------------------------------------------------------
% Quellcodeverzeichnis
% ----------------------------------------------------------------
\clearpage
\phantomsection
\addcontentsline{toc}{chapter}{Quellcodeverzeichnis}
\listoflistings
\clearpage

% ----------------------------------------------------------------
% Tabellenverzeichnis
% ----------------------------------------------------------------
\phantomsection
\addcontentsline{toc}{chapter}{Tabellenverzeichnis}
\listoftables
\clearpage

% ----------------------------------------------------------------
% Abbildungsverzeichnis
% ----------------------------------------------------------------
\phantomsection
\addcontentsline{toc}{chapter}{Abbildungsverzeichnis}
\listoffigures
\clearpage

% ----------------------------------------------------------------
% Literaturverzeichnis
% ----------------------------------------------------------------
\phantomsection
\addcontentsline{toc}{chapter}{Literaturverzeichnis}
\bibliographystyle{ieeetr}
\bibliography{Literaturverzeichnis}

\end{document}