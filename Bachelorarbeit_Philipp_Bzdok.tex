\author{Philipp Dominik Bzdok}

\title{Bachelorarbeit Medientechnologie \linebreak \linebreak
Authentifizierungsmethoden für digitale Fernsignaturen auf Grundlage der Handy-Signatur}

\documentclass[11pt,a4paper,ngerman]{report}
\usepackage[T1]{fontenc}
\usepackage{fontspec}
\usepackage[ngerman]{babel}

\usepackage{fancyhdr}
\pagestyle{fancy}
\fancyhf{}
\setlength{\headheight}{14pt}
\fancyhead[R]{\textsl{\rightmark}}
\fancyfoot[R]{\thepage}

\fancypagestyle{plain}{%
    \renewcommand{\headrulewidth}{0pt}%
    \fancyhf{}%
    \fancyfoot[R]{\thepage}%
}

\usepackage{color}
\usepackage{tabularx}
\usepackage{todonotes}
\usepackage[skip=0pt]{caption}

\usepackage[titles]{tocloft}
\newlistof{listing}{lol}{Quellcodeverzeichnis}

\usepackage[newfloat]{minted}
\usemintedstyle{xcode}
\setminted[java]{linenos,tabsize=2,breaklines,fontsize=\footnotesize,frame=lines,framesep=2ex,autogobble}
\setminted[ruby]{linenos,tabsize=1,breaklines,fontsize=\footnotesize,frame=lines,framesep=2ex,autogobble}

\usepackage{hyperref}
\hypersetup{pdfborder={0 0 0}, breaklinks=true}

\setlength{\parskip}{0.5em}
\setlength{\parindent}{0em}

\newcommand{\ruby}[1]{\mintinline{ruby}{#1}}

% !TEX root = Bachelorarbeit_Philipp_Bzdok

% ----------------------------------------------------------------
% Settings of Quellcodeverzeichnis
% ----------------------------------------------------------------
\setlength{\cftlistingnumwidth}{3em}

\begin{document}
\maketitle

% ----------------------------------------------------------------
% Preamble DE
% ---------------------------------------------------------------- 
\chapter*{Bachelorarbeit}
\textbf{Thema:}

Authentifizierungsmethoden für digitale Fernsignaturen auf Grundlage der Handy-Signatur
\\[4ex]
\textbf{Gutachter:}
\begin{enumerate}
    \item Prof. Dr.-Ing. Luigi Lo Iacono (Technische Hochschule Köln)
    \item Dipl.-Ing. Tobias Wagner (n-design GmbH)
\end{enumerate}
\textbf{Zusammenfassung:}

Was ein Abstract ist wird in der DIN Norm 1426 festgelegt: es ist ein Kurzreferat zur Inhaltsangabe. Die Definition des American National Standards Institute (ANSI) lautet: „An abstract is defined as an abbreviated accurate representation of the con-tents of a document". Es sollten 8-10 Zeilen Text folgen.
\\[4ex]
\textbf{Stichwörter:}

Stichwort, Stichwort (hier sollten maximal 5 Stichwörter folgen)
\\[4ex]
\textbf{Datum:}

\today
\clearpage

% ----------------------------------------------------------------
% Preamble EN
% ---------------------------------------------------------------- 
\chapter*{Bachelors Thesis}
\textbf{Title:}

Authentication methods for digital remote signatures based on mobile phone signatures
\\[4ex]
\textbf{Reviewers:}
\begin{enumerate}
    \item Prof. Dr.-Ing. Luigi Lo Iacono (Technische Hochschule Köln)
    \item Dipl.-Ing. Tobias Wagner (n-design GmbH)
\end{enumerate}
\textbf{Abstract:}

Hier sollte eine Übersetzung des obigen Abstracts auf Englisch erfolgen (und kein kom-plett neuer Text). Auch hier bitte die Begrenzung auf maximal 10 Zeilen Text einhalten.
\\[4ex]
\textbf{Keywords:}

(hier die Übersetzung der obigen Stichworte)
\\[4ex]
\textbf{Date:}

\today
\clearpage

% ----------------------------------------------------------------
% Vorwort
% ---------------------------------------------------------------- 
\chapter*{Vorwort}
Hier möchte ich auf die Umstände der Abschlussarbeit eingehen und denen danken, die mir dabei geholfen haben.
\clearpage

% ----------------------------------------------------------------
% Inhaltsverzeichnis
% ---------------------------------------------------------------- 
\tableofcontents
\clearpage

% ----------------------------------------------------------------
% Inhalt
% ----------------------------------------------------------------

\chapter{Einleitung}
Digitale Signaturen nehmen in der heutigen Zeit eine wichtige Rolle in der Informationstechnik und der Informatik ein. Mithilfe von digitalen Signaturen ist es möglich beliebige Daten, zweifelsfrei Urheberschaft und Integrität nachzuweisen. Die Eigenschaften der Urheberschaft und Integrität sind besonders wichtig im juristischen Kontext der elektronischen Signatur. Der Begriff elektronische Signatur wird oft synonym zur digitalen Signatur verwendet, jedoch ist die elektronische Signatur ein juristischer Begriff, welcher erstmals in einem überarbeiteten Entwurf der EU-Richtlinie 1999/93/EG von der EU-Kommission verwendet wurde.\cite{eSigEU99} Der Begriff elektronische Signatur hingegen stammt aus der Mathematik beziehungsweise aus der Kryptographie.


% ----------------------------------------------------------------
% Quellcodeverzeichnis
% ----------------------------------------------------------------
\clearpage
\phantomsection
\addcontentsline{toc}{chapter}{Quellcodeverzeichnis}
\listoflistings
\clearpage

% ----------------------------------------------------------------
% Literaturverzeichnis
% ----------------------------------------------------------------
\addcontentsline{toc}{chapter}{Literaturverzeichnis}
\bibliographystyle{plain}
\bibliography{Literaturverzeichnis}

\end{document}